\documentclass{beamer}
\usefonttheme[onlymath]{serif}

\title{Monadic Error Handling in Go}
\subtitle{Breaking Beyond Common Go Idioms}
\author{Rebecca Skinner}
\institute{Asteris, LLC}
\date{\today}
\mode<presentation> {\usetheme{Madrid}}

\usepackage[english]{babel}
\usepackage[latin1]{inputenc}
\usepackage{times}
\usepackage[T1]{fontenc}
\usepackage{hyperref}
\usepackage{listings}
\usepackage{color}
\usepackage{amsmath}
\usepackage{csquotes}
\usepackage{verbatim}

\definecolor{comment}{rgb}{145,175,188}
\definecolor{keyword}{rgb}{157,163,199}
\definecolor{string}{rgb}{155,204,174}

\lstset{
  backgroundcolor=\color{white},
  basicstyle=\tiny,
  keywordstyle=\color{keyword},
  commentstyle=\color{comment},
  showspaces=false,
  showstringspaces=false,
  showtabs=false,
  stringstyle=\color{string},
  tabsize=4
}

\newcommand{\chref}[3] {
  {\color{#1} \href{#2}{\underline{#3}}}
}

\AtBeginSection[]{
  \begin{frame}
  \vfill
  \centering
  \begin{beamercolorbox}[sep=8pt,center,shadow=true,rounded=true]{title}
    \usebeamerfont{title}\insertsectionnumber \\ \insertsectionhead\par%
  \end{beamercolorbox}
  \vfill
  \end{frame}
}

\AtBeginSubsection[]{
  \begin{frame}
  \vfill
  \centering
  \begin{beamercolorbox}[sep=8pt,center,shadow=true,rounded=true]{title}
    \usebeamerfont{title}\insertsectionnumber.\insertsubsectionnumber\\\insertsubsectionhead\par%
  \end{beamercolorbox}
  \vfill
  \end{frame}
}

\begin{document}

\begin{frame}
  \titlepage{}
\end{frame}

\section{Introduction}
\begin{frame}
  \frametitle{Let's Write Better Code!}
  Go introduces some novel language features.  The community have
  standardized on some idioms that work, but can we do better?
\end{frame}

\begin{frame}
  \frametitle{Error Handling}
  {\it ``Better Code''} is hard to define in the general case, so let's
  narrow our scope down to error handling and try to define
  {\it better}.
\end{frame}

\subsection{Some Definitions of Better}
\begin{frame}
  \frametitle{Terseness}
  {\bf Terseness}: Using fewer statements.
  \vfill
  Terseness is valuable (up to a point) because every statement is a
  potential error.  As we reduce the number of LOC we reduce the
  expected number of bugs in the application.
  \vfill
  \begin{itemize}
  \item Irrespective of projects defect density, more KLOC == more bugs
  \item Code on the screen is a mental cache
  \end{itemize}
\end{frame}

\begin{frame}
  \frametitle{Expressiveness}
  {\bf Expressiveness}: The amount of meaning in a single statement.
  \vfill
  The more meaning we pack into a statement, the fewer statements we
  have, giving us more terse code.  Expressive code is also easier to
  read because it allows a user to work at a higher level of
  abstraction and be less bogged down in the details.
  \vfill
  \begin{itemize}
    \item More meaning per statement improves terseness
    \item Higher expressiveness makes code easier to understand
  \end{itemize}
\end{frame}

\begin{frame}
  \frametitle{Robustness}
  {\bf Robustness}: Resiliance against mistakes, errors, and changes to the code.
  \vfill
  Robustness in our code means that the structure of our code
  automatically guards against errors.  When we have patterns and
  idioms that help us guard against bugs without having to think about
  it, we're less likely to miss things.
  \vfill
  \begin{itemize}
    \item Prohibiting errors makes code safer
    \item Guaranteed safety makes tests simpler
  \end{itemize}
\end{frame}

\section{Error Handling Idioms}

\begin{frame}[fragile]
  \frametitle{Mutli-Returns}
  One of the most ubiquitious conventions in go programs us using
  multi-return to indicate an error or missing value.  It offers an
  alternative approach to idioms used in other procedural and object
  oriented languages such as:
  \vfill
  \begin{itemize}
  \item Exceptions
  \item Out-Parameters
  \item Error constants (e.g. \verb!errno!)
  \item Returning structs or tuples
  \item Crashing
  \end{itemize}
\end{frame}

\subsection{Examples}
\begin{frame}[fragile]
  \frametitle{Error Handling with Multi-Return}
  \begin{columns}[T]
    \begin{column}{.48\textwidth}
\begin{lstlisting}
value, err = FunctionCall()
if err != nil {
  return 0, err
}
\end{lstlisting}
    \end{column}
    \hfill%
    \begin{column}{.48\textwidth}
\begin{lstlisting}
if result, err := FunctionCall(); err != nil {
  DoTheThing(result)
}
\end{lstlisting}
    \end{column}
  \end{columns}
\end{frame}

\begin{frame}[fragile]
  \frametitle{Mutli-Return with Bool}
  \begin{columns}[T]
    \begin{column}{.48\textwidth}
\begin{lstlisting}
value, ok := dataMap["key1"]
if !ok {
        value = defaultValue
}
\end{lstlisting}
    \end{column}
    \hfill
    \begin{column}{.48\textwidth}
\begin{lstlisting}
if value, ok := dataMap["key2"]; ok {
  doSomething(value)
}
\end{lstlisting}
    \end{column}
  \end{columns}
\end{frame}

\begin{frame}
  \frametitle{What's the problem?}
  Using multi-return in this way makes it difficult to achieve
  terseness because it's not expressive.  It's also less robust,
  because we could forget to check an error at any step.

  \vfill

  Every logical operation we want to perform expands to several
  statements.
\end{frame}

\begin{frame}[fragile]
  \frametitle{A Traditional Approach}
\begin{lstlisting}
func Pipeline(g *graph.Graph, id string, factory *render.Factory) (*Result, error) {
        gen := &pipelineGen{Graph: g, RenderingPlant: factory, ID: id}
        node, ok := g.Get(id)
        if !ok {
                return nil, errors.New(id + " no such node in graph")
        }
        task, err := gen.GetTask(node)
        if err != nil {
                return nil, errors.Wrap(err, "unable to plan")
        }
        errorResult, err := gen.DependencyCheck(task)
        if err != nil {
                return nil, errors.Wrap(err, "unable to plan")
        }
        if errorResult != nil {
                return errorResult, nil
        }
        return gen.PlanNode(task)
}
\end{lstlisting}
\end{frame}

\begin{frame}
  \frametitle{What's The Problem?}
  \begin{itemize}
    \item each call becomes 4 lines of code
    \item it's difficult to visually separate out the functions from the error handling logic
    \item reordering calls can be tricky
    \item testing is hard (we have to add mocks)
  \end{itemize}
\end{frame}


\begin{frame}[fragile]
  \frametitle{What if it could look like this?}
\begin{lstlisting}
func Pipeline(g *graph.Graph, id string, factory *render.Factory) (*Result, error) {
        gen := &pipelineGen{Graph: g, RenderingPlant: factory, ID: id}
        gen.GetTask()
        gen.DependencyCheck()
        gen.PlanNode()
}
\end{lstlisting}

  \begin{itemize}
  \item Clearly defined steps
  \item easy to modify
  \item easy to test
  \end{itemize}
\end{frame}

\section{Content vs Structure}

\begin{frame}
  \frametitle{Content and Structure}
  {\bf Structure:} The shape of the code; the way units of functionality are assembled.

  \vfill

  The key difference between our examples is in how we handle
  structure.  In the first case the structure is verbose, and in the
  second it's completely absent.

  \vfill
\end{frame}

\begin{frame}[fragile]
  \frametitle{Encoding Structure}
\begin{lstlisting}
type ExampleFunc func(interface{}) (interface{}, error)

func ExampleStructure(input interface{},
        func1 ExampleFunc,
        func2 ExampleFunc) (interface{}, error) {
        result1, err := func1(input)
        if err != nil {
                return err
        }
        return func2(result1)
}
\end{lstlisting}
  \vfill
  We could encode our structure in a function that checks the result for us
\end{frame}

\begin{frame}[fragile]
  \frametitle{Using the Example}
\begin{lstlisting}
func ShowExample(input interface{}) (interface{}, error) {
        result, err := ExampleStructure(input, operation1, operation2)
        if err != nil {
                return err
        }
        return operation3(result)
}
\end{lstlisting}
  \vfill
  But we still have the same problem...
\end{frame}

\begin{frame}[fragile]
  \frametitle{If we return the same type that we input...}
\begin{lstlisting}
type ExampleFunc func(interface{}) (interface{}, error)

func BetterExample(f1, f2 ExampleFunc) ExampleFunc {
        return func(input interface{}) (interface{}, error) {
                result1, err := f1(input)
                if err != nil {
                        return nil, err
                }
                return f2(result1)
        }
}

func UseExample(input interface{}) (interface{}, error) {
        f := BetterExample(BetterExmaple(operation1, operation2), operation3)
        return f(input)
}
\end{lstlisting}
\end{frame}

\begin{frame}[fragile]
  \frametitle{But now we're tied to {\tt (interface{}, error)}}

  We can create a custom structure for each place in our code, but
  it's better when we have a common language.

  \begin{itemize}
  \item Things that represent structure
  \item A way to chain operations together
  \item Decoupled from the underlying shape of our data
  \end{itemize}
\end{frame}

\section{Monads}

\begin{frame}
  \frametitle{Monads Formalize Structure}

  Monads formalize structure, and let us program with structure
  separate from the code that the structure is controlling.
\end{frame}

\subsection{Monads in Practice}

\begin{frame}[fragile]
  \frametitle{Monad Interface}
\begin{lstlisting}
type Monad interface {
        AndThen(func(interface{}) Monad) Monad
        Return(i interface{}) Monad
}

func (m Either) AndThen(f func(interface{}) monad.Monad) monad.Monad {
        if m.IsLeft() {
                return m
        }
        inner, ok := m.inner.(RightType)
        if !ok {
                return Left(errors.New("invalid either type"))
        }
        return f(inner.Val)
}

func (m Maybe) AndThen(f func(interface{}) monad.Monad) monad.Monad {
        if asJust, ok := m.internal.(just); ok {
                return f(asJust.Val)
        }
        return m
}
\end{lstlisting}
\end{frame}



\section{Why?}
\begin{frame}
  \frametitle{Precedent}
  Techniques that originated in functional languages are being used to
  great effect in many languages across the spectrum of paradigms

  \begin{itemize}
  \item C$\sharp$
  \item Java
  \item C++
  \item ruby
  \end{itemize}
\end{frame}

\begin{frame}
  \frametitle{Go Is Already Multi-Paradigm}
  Go already supports multiple development paradigms; adding
  functional idioms is not straying from the spirit of the language
\end{frame}

\begin{frame}
  \frametitle{Go Isn't Safe}

\end{frame}

\begin{frame}
  \frametitle{Go Has Language Features We Need}

\end{frame}

\begin{frame}
  \frametitle{It's Helpful}

\end{frame}


\end{document}
