% Copyright 2017 Rebecca Skinner
%
% This work is licensed under the Creative Commons
% Attribution-ShareAlike 4.0 International License. To view a copy of
% this license, visit http://creativecommons.org/licenses/by-sa/4.0/
% or send a letter to Creative Commons, PO Box 1866, Mountain View, CA
% 94042, USA.
\documentclass{beamer}

\title{Introducing Erlang}
\subtitle{A Brief Survey of The Erlang Programming Language}
\author{Rebecca Skinner}
\institute{Rackspace Hosting}
\date{\today}

\mode<presentation> {\usetheme{metropolis}}

\usepackage[english]{babel}
\usepackage{times}
\usepackage[T1]{fontenc}
\usepackage{hyperref}
\usepackage{listings}
\usepackage{color}
\usepackage{amsmath}
\usepackage{csquotes}
\usepackage{verbatim}
\usepackage{fontspec}
\usepackage{pbox}

\definecolor{comment}{rgb}{145,175,188}
\definecolor{keyword}{rgb}{157,163,199}
\definecolor{string}{rgb}{155,204,174}

\lstset{ % add your own preferences
  basicstyle=\small,
  showspaces=false,
  showtabs=false,
  numbers=none,
  numbersep=5pt,
  showstringspaces=false,
  stringstyle=\color[rgb]{0.16, .47, 0},
  tabsize=1
}

\newcommand{\chref}[3] {
  {\color{#1} \href{#2}{\underline{#3}}}
}

\AtBeginSection[]{
  \begin{frame}
    \vfill
    \centering
    \begin{beamercolorbox}[sep=8pt,center,shadow=true,rounded=true]{title}
      \usebeamerfont{title}\insertsectionnumber \\ \insertsectionhead\par%
    \end{beamercolorbox}
    \vfill
  \end{frame}
}

\AtBeginSubsection[]{
  \begin{frame}
    \vfill
    \centering
    \begin{beamercolorbox}[sep=8pt,center,shadow=true,rounded=true]{title}
      \usebeamerfont{title}\insertsectionnumber.\insertsubsectionnumber\\\insertsubsectionhead\par%
    \end{beamercolorbox}
    \vfill
  \end{frame}
}

\begin{document}
\begin{frame}
  \titlepage{}
  \begin{center}
    \small{\chref{blue}{http://creativecommons.org/licenses/by-sa/4.0/}{LICENSE}}
  \end{center}
\end{frame}

\section{What is Erlang?}

\begin{frame}
  \frametitle{Erlang is Weird}
  \begin{itemize}
  \item strong, dynamic typing
  \item Pure functional
  \item Syntax influenced by prolog, lisp, and smalltalk
  \item Compiled to BEAM
  \end{itemize}
\end{frame}

\begin{frame}
  \frametitle{The Beam}
  Beam is name for erlang's virtual machine.  Erlang code compiles to
  beam bytecode.  The beam provides process and thread management, as
  well as networking and IPC across distributed processes.
\end{frame}

\begin{frame}[fragile]
  \frametitle{Beam Languages}
  \begin{itemize}
  \item Elixer: a beam language with syntax superficially resembling ruby, because ruby
  \item LFE: A lisp for the beam, because {\small \verb!(((even (lisp is) 'better) than) (erlang syntax))!}
  \item Idris: a dependently typed ML written in haskell
  \item Purescript: more-or-less haskell, but for javascript (and the beam too, I guess)
  \item Elm (eventually?): megablock training wheels version of purescript.  They keep talking about eventually targetting beam in addition to javascript.
  \end{itemize}
\end{frame}

\begin{frame}
  \frametitle{OTP}
  The Open Telecom Platform (OTP) is a set of libraries and middleware
  to facilitate writing distributed stateful appliations.  OTP
  facilitates the development of highly scalable, fault tolerant
  applications.
\end{frame}

\section{Introducing Erlang}

\begin{frame}[fragile]
  \frametitle{Hello, Erlang}
\begin{lstlisting}[language=erlang]
-module(hello).
-export([hello_name/1]).
hello_name(Name) ->
    io:format("hello, ~p\n", [Name]).
\end{lstlisting}
\end{frame}

\begin{frame}[fragile]
  \frametitle{Modules}
  There is a 1:1 relationship between erlang files and modules.  Each
  module defines it's name using the \verb!-module! preprocessor
  directive, and declares all exported functions with the
  \verb!-export! directive.

  Erlang modules act as namespaces.  When calling a function from
  another module, or a REPL, the module name should be prefixed to the
  function name.

\begin{lstlisting}[language=erlang]
hello:hello_name("World").
\end{lstlisting}
\end{frame}

\begin{frame}
  \frametitle{Variables}
  Erlang variables are immutable and statically typed.  All erlang
  variables start with a capital letter.  Erlang variables can have
  any of the following types, which are assigned at runtime:
  \begin{columns}
    \begin{column}{0.5\textwidth}
      \begin{itemize}
      \item integers
      \item floats
      \item atoms
      \item binary strings
      \item references
      \item functions
      \item ports
      \end{itemize}
    \end{column}
    \begin{column}{0.5\textwidth}
      \begin{itemize}
      \item processes
      \item tuples
      \item maps
      \item lists
      \item strings
      \item records
      \item booleans
      \end{itemize}
    \end{column}
  \end{columns}
\end{frame}

\begin{frame}
  \frametitle{Terms}
  The fundamental type in erlang is the \emph{term}.  A term is the
  fundamental representation of a value in erlang.
\end{frame}

\begin{frame}
  \frametitle{Atoms}
  An erlang atom is similar to a symol in lisp or ruby.  Atoms can be
  used just like you might use an enum or named constant in another
  language.  Erlang uses atoms to name records, and they can also be
  used to identify processes or messages.
\end{frame}

\begin{frame}
  \frametitle{Lists}
  Lists in erlang are 1-indexed heterogenous collections of terms.
  Erlang lists are constructed of cons cells.  As a matter of
  convenience, an erlang list may be constructed as a sequence of
  terms without specifying the cons cells.  A list shown with explicit
  cons cells is known as it's \emph{canonical representation}.
\end{frame}

\begin{frame}[fragile]
  \frametitle{List Examples}
\begin{lstlisting}[language=erlang]
> %% integer, string, atom
> List1 = [1,"two", three].
[1,"two",three]
> %% get the first element from the list
> lists:nth(List1,1).
1.
> List1 == [1 | "two" | [three | []]].
true
\end{lstlisting}
\end{frame}

\begin{frame}[fragile]
  \frametitle{Strings and Characters}
  Erlang does not differentiate between integer values and characters.
  All characters are stored as integers representing the UTF8
  codepoint of the character.  A string is an integer list of
  codepoints.

  Character literals can be written by prefixing the characteer with
  the \textdollar{} symbol.
\begin{lstlisting}[language=erlang]
[$h,$e,$l,$l,$o].
"hello"
$a.
97
\end{lstlisting}
\end{frame}

\begin{frame}
  \frametitle{Bit Strings}
  Erlang has special syntax for dealing with packed binary data.  This
  feature makes it particularly suited for dealing with low level
  network protocols.  A bit string is a packed binary representation
  of a list of binary values.  Lists of integers can be passed into a
  bit string to be packed into their binary representation.  Like with
  lists, the \textdollar{} symbol can be used to insert the numeric
  value of a character literal.
\end{frame}

\begin{frame}[fragile]
  \frametitle{Bit Strings Cont.}
  When creating a bit string, there are several options that can be
  provided to specify in detail how the binary value will be packed.
  The general format for a field in a bit string is:

  \verb!Value:Size/Type-Specifier-List!

  Where \emph{Size} is the number of bits used the represent the
  value, and \emph{Type-Specifier-List} is a hyphen separated list of
  options.
\end{frame}

\begin{frame}[fragile]
  \begin{tabular}{|l|p{3in}|}
    \hline
    Type Specifier & Description \\
    \hline
    \emph{type} & The type of the term.  Must be one of: \emph{integer, float, binary, bytes, bitstring, bits, utf8, utf16, utf32} \\
    \hline
    \emph{endianness} & The endianness of the value: Must be one of \emph{big, little, native} \\
    \hline
    \emph{unit} & The unit size of a binary value.  Fields will be stored in $size * unit$ bit sets, and unsized fields will be aligned to \emph{Unit} bits. \\
    \hline
    \emph{Signedness} & Whether or not the value should be considered signed (for pattern matching).  Must be one of \emph{signed, unsigned}\\
    \hline
  \end{tabular}
\end{frame}

\begin{frame}[fragile]
  \frametitle{Bit String Examples}
\begin{lstlisting}[language=erlang]
IPHeader = <<Version:4,5/integer-unit:4,DSCP:6,
             ECN:2,Len:16/big,
             Ident:16,0/integer-unit:1,
             DF/integer-unit:1,MF/integer-unit:1,
             FO:24, TTL,Proto,Checksum:16,
             SourceIP:32/big,
             DestIP:32/big>>
\end{lstlisting}
\end{frame}

\begin{frame}[fragile]
  \frametitle{Pattern Matching}
  Erlang supports pattern matching of variables.  Record and tuple
  fields, list elements, and fields from bit strings can all be
  pattern matched.
\begin{lstlisting}[language=erlang]
[A, B, 3] = [1, 2, 3].
{foo, Bar} = {foo, bar}.
\end{lstlisting}
\end{frame}

\begin{frame}
  \frametitle{Functions}
  Functions in erlang are pure.  They may accept any number of
  arguments, and always return exactly one value.  The return value of
  an erlang function is the value of the last statement in it's
  expression.

  Statements within a function are separated by commas, and a function
  is terminated with a period.
\end{frame}

\begin{frame}[fragile]
  \frametitle{Basic Function Example}
\begin{lstlisting}[language=erlang]
show_addition(A, B) ->
    Sum = A + B,
    {A, B, Sum}.
\end{lstlisting}
\end{frame}

\begin{frame}
  \frametitle{Function Arity}
  The arity of the function is the number of arguments the function
  accepts as input.  In Erlang, the arity of a function is part of the
  fully qualified name of that function.  The arity of a function may
  be omitted when calling the function, otherwise arity of a function
  is specified by a forward slash and the arity.

  Erlang functions are not polymorphic over the number of arguments
  per-se, but due to the fact that the arity of the function is part
  of the functions canonical name, functions of different arities may
  share a name.
\end{frame}

\begin{frame}[fragile]
  \frametitle{Function Arity Example}
\begin{lstlisting}[language=erlang]
add(A,B) ->
    A + B.
add(A,B,C) ->
    A + B + C.
get_add_function(Count) ->
    case Count of
        Count = 2 -> fun add/2;
        Count = 3 -> fun add/3;
        _ -> {error, "invalid count"}
    end.
\end{lstlisting}
\end{frame}

\begin{frame}
  \frametitle{First Class Functions}
  Erlang functions can be passed around as regular values.  They can
  be stored in lists, maps, and records, passed in and returned from
  functions, and sent as messages to other processes.  The \emph{fun}
  keyword is used to refer to a function value.
\end{frame}

\begin{frame}[fragile]
  \frametitle{Another Function Example}
\begin{lstlisting}[language=erlang]
foldFunc(Elem, Carry) ->
  BindFunction = monad:bind(maybe_m:maybe_monad()),
  BindFunction(Carry,
    fun(BareCarry) ->
        BindFunction(Elem,
          fun(BareElem) ->
              maybe_sum(BareCarry, BareElem)
          end
        )
    end
\end{lstlisting}
\end{frame}

\begin{frame}
  \frametitle{Guards and Pattern Matches.}
  A function can specify limits to the input values it operates on.

  Pattern matching may be used to limit the input values of a function
  at a structural level.  Erlang supports unification of pattern
  matched values, allowing for equality comparisions along with
  destructuring.

  In addition to pattern matching \emph{where} statement creates a
  guard that allows a developer to specify a set of contraints on the
  values of function parameters, including the values of variables
  assigned to destructured elements of an input value.

  Piecewise functions differentiated by pattern matching and guards
  should be separated by semicolons.
\end{frame}

\begin{frame}[fragile]
  \frametitle{Pattern Matching Example}
\begin{lstlisting}[language=erlang]
lists_eq([],[]) ->
    true;
lists_eq([X|XS],[X|YS]) ->
    lists_eq(XS,YS);
lists_eq(_, _) ->
    false.
\end{lstlisting}

  Note how, in this example, the same variable, \emph{X} is used to
  represent the head of both lists.  Thanks to unification this
  pattern match allows us to assert equivalence over the values at the
  head of each list, leading to a very terse method for comparing
\end{frame}

\begin{frame}[fragile]
  \frametitle{Guard Examples}
\begin{lstlisting}[language=erlang]
fib(X) when (X =< 1) ->
    1;
fib(X) ->
    fib(X - 1) + fib(X - 2).
\end{lstlisting}
\end{frame}

\begin{frame}[fragile]
  \frametitle{Tuples}
  Erlang suports heterogenous tuples of arbitrary size.  Erlang uses
  braces to reprsent tuples.  Elements of a tuple may be accessed
  through destructuring.

  \verb!{tuple1, tuple2, tuple3}!
\end{frame}

\begin{frame}[fragile]
  \frametitle{Maps}
  Erlang maps are sets of key/value associations.  The \verb!#! symbol
  is used to define a map.  Keys in an erlang map may be any term,
  including functions, processes, maps, or, most commonly, atoms.

\begin{lstlisting}[language=erlang]
mapTest() ->
    F = fun () -> 1 end,
    {F, #{F => foo, foo => bar}}.
\end{lstlisting}
\end{frame}

\begin{frame}[fragile]
  \frametitle{Destructuring Maps}
  Maps may be destructured using the \verb!:=! operator.  A
  destructured map over a pattern must contain at least the specified
  elements, but the pattern need not include every element of the map.

\begin{lstlisting}[language=erlang]
mapTest(#{foo := Value, const := const}) -> Value.

% okay, has keys foo and const
demo:mapTest(#{bar => bar, foo => foo, const => const}).

% not okay, no key 'const'
demo:mapTest(#{bar => bar, foo => foo, c => const}).
\end{lstlisting}
\end{frame}

\begin{frame}[fragile]
  \frametitle{Updating Maps}
  Like all values in erlang, maps are immutable.  Erlang provides some
  syntactic sugar to make it easier to add and update values in maps.
\begin{lstlisting}[language=erlang]
mapTest() ->
    M = #{foo => foo},
    WithBar = M#{bar => bar},
    Updated = WithBar#{foo := foo1},
    Updated#{baz => baz}.

mapTest(). % #{bar => bar,baz => baz,foo => foo1}
\end{lstlisting}
\end{frame}

\section{OTP}

\begin{frame}
  \frametitle{What is OTP?}
  The Open Telecom Platform (OTP) is a set of libraries, middleware,
  and tools designed to assist in building distributed fault tolerant
  applications.  OTP is part of the erlang distribution and runtime,
  but can be used with other Beam languages such as Elixer.
\end{frame}

\begin{frame}
  \frametitle{OTP Architecture}
  OTP prescribes an architecture that lends itself particularly well
  to fault tolerant distributed applications, and plays to the
  particular strengths of ther Beam and erlang as a language.  OTP is
  built around the concept of message passing, actors, and
  \emph{supervision trees}.
\end{frame}

\begin{frame}
  \frametitle{Behaviors}
  Erlang behaviors are, essentially, a module level interface.  A
  behavior defines a set of required functions and their arity.  OTP
  defines several behaviors that are important for the OTP
  Architecture.
\end{frame}

\begin{frame}
  \frametitle{OTP Behaviors}
  \begin{tabular}{|l|r|}
    \hline
    \emph{Behavior} & \emph{Description} \\
    \hline
    {\tt supervisor} & Monitors workers and other supervisors \\
    \hline
    {\tt supervisor\_bridge} & Supervises non-OTP applications \\
    \hline
    {\tt gen\_server} & A generic server \\
    \hline
    {\tt gen\_fsm} & A generic finite state machine \\
    \hline
    {\tt gen\_event} & A generic event handler \\
    \hline
    {\tt gen\_statem} & A generic state machine \\
    \hline
  \end{tabular}
\end{frame}

\begin{frame}
  \frametitle{Supervisors}
  A supervisor is one of the most common types of OTP application
  processes.  Supervisors are simple applications whose job is manage
  the lifecycle of
\end{frame}

\begin{frame}[fragile]
  \frametitle{Sample Supervisor}
\begin{lstlisting}[language=erlang]
start_link() ->
    supervisor:start_link({local, ?SERVER}, ?MODULE, []).
init([]) ->
    SupFlags = #{strategy => one_for_one,
                 intensity => 1,
                 period => 5},
    AChild = #{id => 'AName',
               start => {'AModule', start_link, []},
               restart => permanent,
               shutdown => 5000,
               type => worker,
               modules => ['AModule']},
    {ok, {SupFlags, [AChild]}}.
\end{lstlisting}
\end{frame}

\begin{frame}
  \frametitle{Supervisor Restart Strategies}
    \begin{tabular}{|l|p{2.5in}|}
      \hline
      strategy & description \\
      \hline
      {\tt one\_for\_one} & if a child process terminates, restart it \\
      \hline
      {\tt one\_for\_all} & if a child process terminates, terminate all other children then restart them all \\
      \hline
      {\tt rest\_for\_one} & if a child process terminates, all processes started after the child are also terminated, then all terminated processes are restarted \\
      \hline
      {\tt simple\_one\_for\_one} & a variant of {\tt one\_for\_one} where all child processes are dynamically created instances of a single process \\
      \hline
    \end{tabular}
\end{frame}

\section{Tooling}

\begin{frame}
  \frametitle{Kerl}
  \href{https://github.com/kerl/kerl}{kerl} is a utility that allows
  you to manage installations of erlang. Kerl will compile and install
  local versions of erlang, and provides bash scripts to allow you to
  activate and deactivate environments to use each version.  Kerl also
  allows you to install a per-project OTP distribution.
\end{frame}

\begin{frame}
  \frametitle{Rebar3}
  \href{https://github.com/erlang/rebar3}{rebar3} is a tool to allow
  you to manage packages and build erlang applications.  Rebar3
  expects conventional OTP structured applications, and provides tools
  to build and package erlang applications with their runtime.
\end{frame}

\begin{frame}
  \frametitle{Dialyzer}
  \href{http://erlang.org/doc/man/dialyzer.html}{dialyzer} is an
  optional success typing system that allows you to perform
  rudimentary static analysis of erlang applications.  Dialyzer is
  based on
  \href{http://user.it.uu.se/~kostis/Papers/war_story.pdf}{published
    research} that investigated concrete real world applications for
  gradual success typing to improve the reliability of applications
  written in dynamically typed languages.
\end{frame}

\begin{frame}
  \frametitle{Observer}
  Observer provides debugging and observability to erlang
  applications.  Observer allows you to connect to remote beam
  instances, view individual process mailboxes, send messages, delete
  messages, visualize the supervisor tree, and to kill and restart
  supervisors and worker processes.
\end{frame}

\section{Questions?}

\end{document}
