% Copyright 2017 Rebecca Skinner
%
% This work is licensed under the Creative Commons
% Attribution-ShareAlike 4.0 International License. To view a copy of
% this license, visit http://creativecommons.org/licenses/by-sa/4.0/
% or send a letter to Creative Commons, PO Box 1866, Mountain View, CA
% 94042, USA.
\documentclass{beamer}

\title{Building Tech Talks}
\subtitle{Who, What, When, and Where of Talking At People About Tech}
\author{Rebecca Skinner}
\institute{Rackspace Hosting}
\date{\today}

\mode<presentation> {\usetheme{metropolis}}

\usepackage[english]{babel}
\usepackage{times}
\usepackage[T1]{fontenc}
\usepackage{hyperref}
\usepackage{listings}
\usepackage{color}
\usepackage{amsmath}
\usepackage{csquotes}
\usepackage{verbatim}
\usepackage{fontspec}
\usepackage{pbox}

\definecolor{comment}{rgb}{145,175,188}
\definecolor{keyword}{rgb}{157,163,199}
\definecolor{string}{rgb}{155,204,174}

\lstset{ % add your own preferences
  basicstyle=\small,
  showspaces=false,
  showtabs=false,
  numbers=none,
  numbersep=5pt,
  showstringspaces=false,
  stringstyle=\color[rgb]{0.16, .47, 0},
  tabsize=1
}

\newcommand{\chref}[3] {
  {\color{#1} \href{#2}{\underline{#3}}}
}

\AtBeginSection[]{
  \begin{frame}
    \vfill
    \centering
    \begin{beamercolorbox}[sep=8pt,center,shadow=true,rounded=true]{title}
      \usebeamerfont{title}\insertsectionnumber \\ \insertsectionhead\par%
    \end{beamercolorbox}
    \vfill
  \end{frame}
}

\AtBeginSubsection[]{
  \begin{frame}
    \vfill
    \centering
    \begin{beamercolorbox}[sep=8pt,center,shadow=true,rounded=true]{title}
      \usebeamerfont{title}\insertsectionnumber.\insertsubsectionnumber\\\insertsubsectionhead\par%
    \end{beamercolorbox}
    \vfill
  \end{frame}
}

\begin{document}
\begin{frame}
  \titlepage{}
  \begin{center}
    \small{\chref{blue}{http://creativecommons.org/licenses/by-sa/4.0/}{LICENSE}}
  \end{center}
\end{frame}

\section{Introduction}
\begin{frame}
  \frametitle{Meta, Meta Meta}
  \includegraphics[width=.85\paperwidth]{images/abed.png}
\end{frame}
\begin{frame}
  \frametitle{About This Talk}
  \begin{itemize}
  \item I Don't Have Anything to Say
  \item It Takes Too Much Time
  \item How Can I Explain It in 60 Minutes? 45? 30?
  \end{itemize}
\end{frame}
\begin{frame}[fragile]
  \frametitle{Some Talk Guidelines}
  \begin{itemize}
  \item 1 hour of prep per minute of speaking
  \item 0.5 to 1 slide per minute fo speaking
  \item $\frac{1}{3}$rd of your content will be recalled
  \end{itemize}
\end{frame}

\section{Selecting a Talk}

\subsection{What to talk about}
\begin{frame}
  \frametitle{Picking a Topic}
  You don't need to be an expert on something to speak about- in fact
  some times expertise hurts more than it helps.  So how do you know
  what to talk about?
  \begin{itemize}
  \item Tell a story
  \item Bring a unique perspective
  \item Empathize with people a little behind you
  \end{itemize}
\end{frame}

\begin{frame}
  \frametitle{The Hero's Journey}
  \includegraphics[width=.85\paperwidth]{images/ww.jpg}
\end{frame}

\begin{frame}
  \frametitle{Telling a Story}
  A good talk should tell a story.  In this talk we're telling the
  story of a speaker.  Find the story that fits your topic, and make
  the listeners the hero of the talk.
\end{frame}

\begin{frame}
  \frametitle{Empathize With People}
  You don't have to be an expert to have a good talk.  There's a lot
  of value in helping people just a little behind you catch up to
  where you are.
\end{frame}

\subsection{Who to talk to}

\begin{frame}
  \frametitle{Finding Your Audience}
  Don't try to speak to every single person in the audience.  Instead,
  pick one or two archteypical people and speak to them.  A good
  audience to pick is:
  \begin{itemize}
  \item The complete beginner
  \item You, 6 to 12 months ago
  \item The pedantic expert
  \end{itemize}
\end{frame}

\subsection{Having Opinions}

\begin{frame}
  \frametitle{Speak With a Purpose}
  When you're deciding what to say in a talk, it's better to speak to
  a narrow purpose than to try to give a broad overview.
\end{frame}

\begin{frame}
  \frametitle{Specificity}
  Talks aren't lectures.  You don't have a semester, a book, or even a
  full blog post of reader attention.  Keep your focus on one narrow
  thing.
  \begin{itemize}
  \item ``Building a parser with haskell'', not, ``An introduction to haskell''
  \item ``How to build a tech talk'', not, ``How to speak publicly and teach people things''
  \item ``Configuring your local dev environment with Converge'' not ``The theory of configuration management''
  \end{itemize}
\end{frame}

\section{Pedagogy and Epistemology}
\begin{frame}
  \frametitle{Picking A Path}
  \includegraphics[width=.85\paperwidth]{images/graph.png}
\end{frame}

\subsection{The Important Bits}

\begin{frame}
  \frametitle{Figure Out What You Want To Teach}
  Pick a few short bullet points.  No more than one per 20 minutes of
  your talk.  Draw them out.
\end{frame}

\begin{frame}
  \frametitle{Figure Out What You Can Skip}
  Pruning is the hardest part of putting together a presetation..
  Relentlessly prune away anything that isn't important to your story.
\end{frame}


\begin{frame}
  \frametitle{Theory of Mind}
  Be considerate of what your listeners already know.  Pick a few
  things that you think they'll know already as touch points to start
  your presentation.  Skip everything else they know already, and
  everything they can make an easy logical leap to understanding based
  on what you're already talking about.
\end{frame}

\subsection{SWBAT}

\begin{frame}
  \frametitle{Students Will Be Able To}
  Keep in mind what you want the audience to get to your talk.  Every
  slide should further their journy to that point.  Start with the end
  of the presetation, or the end of each section.  Work backwards,
  each slide should be because it will help the audience learn, do, or
  think about something.
\end{frame}


\begin{frame}
  \frametitle{Create a Digraph}
  Figure out what concepts you need to understand to understand the
  things that are important.  What needs to be learned in what order.
\end{frame}

\begin{frame}
  \frametitle{A-Star is Born}
  Pick a path through the graph.  The nodes are your sections or
  headings.  The edges are your slides.
\end{frame}


\subsection{Seven Bridges of Königsberg}

\begin{frame}
  \frametitle{Seven Bridges}
  \includegraphics[width=.85\paperwidth]{images/bridges.jpg}
\end{frame}

\begin{frame}
  \frametitle{Connecting Ideas}
  You should find one or two ideas that are your ``base of
  operations''.  These will be touch points that you keep coming back
  to.  In this presentation, it's how to structure your ideas.
\end{frame}

\begin{frame}
  \frametitle{Each Concept is a Bridge}
  Consider the euler circuit.  Each new idea is a bridge.  You want to
  lead the audience across new ideas, and you don't want to backtrack.
  Make sure you can still end up on your keystone ideas.
\end{frame}


\section{Slides}

\begin{frame}
  \frametitle{Slides}
  \includegraphics[width=.85\paperwidth]{images/slide.jpg}
\end{frame}

\subsection{Beamer Me Up Scotty}
\begin{frame}
  \frametitle{Find a Template and Stick With It}
  Beamer is a very fast way to put together nice looking slides.
  Finding a tool that lets you put together ideas quickly can greatly
  reduce the time and energy investment in putting together slides.
\end{frame}

\subsection{Literate Content}
\begin{frame}
  \frametitle{Literate Programming For Code Slides}
  Tools like orgmode allow you to integrate your source code, output,
  and slide data into one document.  Look at ways to create slides
  from tools like org-mode, jupyter and mathematica notebooks, etc.
\end{frame}

\subsection{Gene Simmons Dummy}

\begin{frame}
  \frametitle{KISS}
    \includegraphics[width=.85\paperwidth]{images/kiss.jpg}
\end{frame}

\begin{frame}
  \frametitle{KISS}
  Remember, in the end, to Keep It Simple, Simmons.
\end{frame}

\section{Questions?}

\end{document}
