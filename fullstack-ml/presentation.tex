% Copyright 2017 Rebecca Skinner
%
% This work is licensed under the Creative Commons
% Attribution-ShareAlike 4.0 International License. To view a copy of
% this license, visit http://creativecommons.org/licenses/by-sa/4.0/
% or send a letter to Creative Commons, PO Box 1866, Mountain View, CA
% 94042, USA.
\documentclass{beamer}

\title{Building Web Apps with Haskell and Elm}
\author{Rebecca Skinner\\ \small{@cercerilla}}
\date{\today}

\mode<presentation> {\usetheme{metropolis}}

\usepackage[english]{babel}
\usepackage{times}
\usepackage[T1]{fontenc}
\usepackage{hyperref}
\usepackage{listings}
\usepackage{color}
\usepackage{amsmath}
\usepackage{csquotes}
\usepackage{verbatim}
\usepackage{fontspec}
\usepackage{pbox}

\definecolor{comment}{rgb}{145,175,188}
\definecolor{keyword}{rgb}{157,163,199}
\definecolor{string}{rgb}{155,204,174}

\lstset{ % add your own preferences
  basicstyle=\tiny,
  showspaces=false,
  showtabs=false,
  numbers=none,
  numbersep=5pt,
  showstringspaces=false,
  stringstyle=\color[rgb]{0.16, .47, 0},
  tabsize=1
}

\newcommand{\chref}[3] {
  {\color{#1} \href{#2}{\underline{#3}}}
}

\AtBeginSection[]{
  \begin{frame}
    \vfill
    \centering
    \begin{beamercolorbox}[sep=8pt,center,shadow=true,rounded=true]{title}
      \usebeamerfont{title}\insertsectionnumber \\ \insertsectionhead\par%
    \end{beamercolorbox}
    \vfill
  \end{frame}
}

\AtBeginSubsection[]{
  \begin{frame}
    \vfill
    \centering
    \begin{beamercolorbox}[sep=8pt,center,shadow=true,rounded=true]{title}
      \usebeamerfont{title}\insertsectionnumber.\insertsubsectionnumber\\\insertsubsectionhead\par%
    \end{beamercolorbox}
    \vfill
  \end{frame}
}

\begin{document}
\begin{frame}
  \titlepage{}
  \begin{center}
    \small{\chref{blue}{http://creativecommons.org/licenses/by-sa/4.0/}{LICENSE}}
  \end{center}
\end{frame}

\section{Introduction}

\begin{frame}
  \frametitle{Experience Report}
  This presentation will focus on the experience of using these two
  technologies together to build a working application.
\end{frame}

\begin{frame}
  \frametitle{MetaLanguage, Not Machine Learning}
  ML stands for \emph{MetaLanguage}.  It's a language originally
  designed for working with the LCF theorem prover. Today, there are
  quite a few ML dialects, as well as several languages that have been
  influenced by ML
\end{frame}

\begin{frame}
  \frametitle{Haskell}
  Haskell is a functional programming language with a syntax that is
  heavily influenced by the style of ML dialects.  Haskell compiles to:
  \begin{itemize}
  \item x86
  \item ARM
  \item LLVM
  \item Web Assembly
  \item Javascript
  \end{itemize}
\end{frame}

\begin{frame}
  \frametitle{Elm}
  Elm is a functional language written in haskell and built for
  creating client-side single-page applications.  Elm currently
  compiles to:
  \begin{itemize}
  \item Javascript
  \end{itemize}
\end{frame}

\section{Motivation}

\begin{frame}
  \frametitle{Make Programs that Don't Go Wrong}
  Debugging is much harder than writing code, so I would like the
  computer to help me find my errors early.
\end{frame}

\begin{frame}
  \frametitle{Make Refactoring Easier}
  For long-lived applications, code is refactored much more often than
  it's written.  I want to work with tools that make refactoring
  easier and less error-prone.
\end{frame}

\begin{frame}
  \frametitle{Automate Away Tests}
  I want the biggest return on investment with automated checking.  I
  want to have to write as few tests as possible to ensure my programs
  work well.
\end{frame}

\begin{frame}
  \frametitle{Ergonomics}
  ML-like languages have comfortable ergonomics that make them nicer
  to work with compared to languages like Javascript.
\end{frame}

\subsection{Non-Motivators}

\begin{frame}
  \frametitle{Popularity}
  I don't care how many people already know my implementation language.
\end{frame}

\begin{frame}
  \frametitle{Performance}
  Raw performance is less important than good-enough performance along
  with correctness and ease of refactoring.
\end{frame}

\begin{frame}
  \frametitle{Library Support}
  I only care about the libraries I need for my project, and not about
  the totality of the library ecosystem.
\end{frame}

\section{Building A Server}

\begin{frame}
  \frametitle{Haskell for Backend Services}
  Haskell is a great choice for building web services.  In particular,
  it has reasonable performance, good libraries, and can generally
  runs without crashing.
\end{frame}

\begin{frame}
  \frametitle{An Embarassement of Frameworks}
  There are several web frameworks for haskell.  My favorites are
  Scotty and Servant.
  \begin{itemize}
    \item Scotty
    \item Spock
    \item Yesod
    \item Happstack
    \item Servant
  \end{itemize}
\end{frame}

\begin{frame}[fragile]
  \frametitle{Beam Me Up, Scotty}
\begin{lstlisting}[language=haskell]
main = scotty 8080 $ do
  S.get "/" defaultFile
  S.post "/:render" $ do
    outputFormat <- param "render"
    inputFormat  <- param "format"
    toConvert    <- getPostBody
    case convertDoc inputFormat outputFormat toConvert of
      Left errMsg -> do
        status status500
        finish
      Right result -> text result
\end{lstlisting}
\end{frame}

\begin{frame}
  \frametitle{The Toolchain}
  \begin{itemize}
  \item nix
  \item stack
  \item docker
  \item quickcheck
  \item hlint
  \end{itemize}
\end{frame}

\begin{frame}
  \frametitle{If It Compiles, It Works}
  Haskell's type system helps find errors for us that we might miss in
  other languages.  This can be a big help when we're prototyping
  because it can save us from having to write a lot of tests or
  spending a lot of time debugging.
\end{frame}

\begin{frame}[fragile]
  \frametitle{QuickCheck}
  Once we decide to write tests, QuickCheck is a great way to write
  many test cases very quickly and easily.
\begin{lstlisting}[language=haskell]
test_wordWrap :: Spec
test_wordWrap = do
  it "returns words in the right order" $ property $
    \str' (Positive width) ->
      let str = BS.pack str'
          words =  concatMap BS.words $ HCat.wordWrap width str
      in words `shouldBe` (BS.words str)
  it "does not break a word" $ property $
    \(Positive width) ->
      let width' = width + 5
          str = BS.pack $ replicate width' 'a'
          wrapped = HCat.wordWrap width str
      in do
        (length wrapped) `shouldBe` 1
        (head wrapped) `shouldBe` str
  it "breaks lines to the appropriate length" $ property $
    \str' (Positive width) ->
      let str = BS.pack str'
          lines = HCat.wordWrap width str
      in shouldSatisfy lines $ all $ \line ->
          (BS.length line) <= width ||
          (length . BS.words $ line) == 1
\end{lstlisting}
\end{frame}

\section{Building A Frontend in Elm}

\begin{frame}
  \frametitle{Why Elm?}
  Elm borrows syntax from Haskell while being narrowly focused on
  providing a useful language for the frontend.
\end{frame}

\begin{frame}
  \frametitle{The Elm Architecture}
  The Elm Architecture is the control flow that elm applications
  follow.  Elm applications are functional reactive event loops that
  should look familiar to you if you've worked with frameworks like
  react.js
\end{frame}

\begin{frame}
  \frametitle{The Differences Between Elm and Haskell}
  Elm is a simpler language that is focused on ease of use,
  approachability, and cohesiveness.  It's less flexible, and offers
  fewer features than Haskell, but makes up for that with a consistent
  and coherent sandbox for developing SPAs.
\end{frame}

\section{A Survey of Other Options}

\begin{frame}
  \frametitle{Backend Languages}
  \begin{itemize}
  \item Go
  \item Rust
  \item Erlang
  \item Ruby
  \item Python
  \item Javascript
  \end{itemize}
\end{frame}

\begin{frame}
  \frametitle{Frontend Languages}
  \begin{itemize}
  \item PureScript
  \item Haskell
  \item Javascript
  \item TypeScript
  \end{itemize}
\end{frame}

\section{Demo}

\end{document}
