% Copyright 2017 Rebecca Skinner
%
% This work is licensed under the Creative Commons
% Attribution-ShareAlike 4.0 International License. To view a copy of
% this license, visit http://creativecommons.org/licenses/by-sa/4.0/
% or send a letter to Creative Commons, PO Box 1866, Mountain View, CA
% 94042, USA.
\documentclass{beamer}

\title{Introducing Go}
\subtitle{A Brief Survey of The Go Programming Language}
\author{Rebecca Skinner}
\institute{Rackspace Hosting}
\date{\today}

\mode<presentation> {\usetheme{metropolis}}

\usepackage[english]{babel}
\usepackage{times}
\usepackage[T1]{fontenc}
\usepackage{hyperref}
\usepackage{listings}
\usepackage{listings-golang}
\usepackage{color}
\usepackage{amsmath}
\usepackage{csquotes}
\usepackage{verbatim}
\usepackage{fontspec}
\usepackage{pbox}

\definecolor{comment}{rgb}{145,175,188}
\definecolor{keyword}{rgb}{157,163,199}
\definecolor{string}{rgb}{155,204,174}

\lstset{ % add your own preferences
  basicstyle=\small,
  showspaces=false,
  showtabs=false,
  numbers=none,
  numbersep=5pt,
  showstringspaces=false,
  stringstyle=\color[rgb]{0.16, .47, 0},
  tabsize=1
}

\newcommand{\chref}[3] {
  {\color{#1} \href{#2}{\underline{#3}}}
}

\AtBeginSection[]{
  \begin{frame}
    \vfill
    \centering
    \begin{beamercolorbox}[sep=8pt,center,shadow=true,rounded=true]{title}
      \usebeamerfont{title}\insertsectionnumber \\ \insertsectionhead\par%
    \end{beamercolorbox}
    \vfill
  \end{frame}
}

\AtBeginSubsection[]{
  \begin{frame}
    \vfill
    \centering
    \begin{beamercolorbox}[sep=8pt,center,shadow=true,rounded=true]{title}
      \usebeamerfont{title}\insertsectionnumber.\insertsubsectionnumber\\\insertsubsectionhead\par%
    \end{beamercolorbox}
    \vfill
  \end{frame}
}

\begin{document}
\begin{frame}
  \titlepage{}
  \begin{center}
    \small{\chref{blue}{http://creativecommons.org/licenses/by-sa/4.0/}{LICENSE}}
  \end{center}
\end{frame}

\section{What is Go}

\begin{frame}
  \frametitle{History of Go}
  Go was announced in 2009 by google.  It is a compiled, statically
  typed, garbage collected language in the C family.  Go was designed
  to provide modern memory management, concurrency models, and
  networking in a lightweight, readable, and performant language.
\end{frame}

\begin{frame}
  \frametitle{Go Today}
  Go has been steadily gaining a foothold.  Although originally
  designed to be a useful systems language, it has not gained large
  scale adoption there compared to languages like Rust or people
  opting to continue using C and C++.  Go's largest inroads have been
  in devops tooling, such as Kubernetes, Terraform, and Telegraf,
  well as in web application backends and middleware.
\end{frame}

\section{The Basics}

\begin{frame}
  \frametitle{10,000 Foot Overview}
  At a very high level Go, as a language, can be described with the
  following features and design choices:
  \begin{itemize}
  \item Compiled
  \item Type-Inferred
  \item Garbage Collected
  \item Statically Typed
  \item Concurrent
  \item Opinionated
  \item Single-Dispatch Object Oriented
  \end{itemize}
\end{frame}

\subsection{Variables}

\begin{frame}
  \frametitle{Variables in Go}
  Variables in go work much like they do in other procedural and
  object oriented languages.  Variables are mutable unless defined as
  constant.  Go differentiates between creating and assigning
  variables.  Creating a variable that already exists, or assigning
  one that doesn't, are both errors.
\end{frame}

\begin{frame}
  \frametitle{Scope}
  Variables in go are block-scoped and closed over the enclosed
  contexts.  A variable in an inner context may shadow a variable in
  the outer context.
\end{frame}

\begin{frame}[fragile]
  \frametitle{Exporting Variables}
  Go uses the concept of \emph{exported} and \emph{unexported}
  variables.  Variables are exported or unexported at the package
  level.  Variable names start with a capital letter are exported, and
  accessible outside of the current package.

\begin{lstlisting}[language=Golang]
package foo

var Exported = 12
var unexported = "foo"
\end{lstlisting}
\end{frame}

\begin{frame}
  \frametitle{Creating Variables}
  Go provides two methods for creating variables.  The {\tt var}
  keyword allows you to define one or more variables in a function or
  package.  The {\tt :=} operator allows to define and initialize a
  new variable inside of a function.
\end{frame}

\begin{frame}
  \frametitle{Default Values}
  Go variables have a default ``zero'' value.  Any newly created
  variables not explicitly given a value default to the zero value.
  The default value of a struct is a struct where each field is set to
  it's default value.  The default value of a pointer is {\tt nil}.
\end{frame}

\begin{frame}[fragile]
  \frametitle{Creating Variables with {\tt var}}
  The var keyword can be used inside or outside of a function to
  create one or more variables.
\begin{lstlisting}[language=Golang]
var x int
var (
  foo  = 10
  bar  = 90.7
  baz  = "threeve!"
  buzz string
)
\end{lstlisting}
\end{frame}

\begin{frame}[fragile]
  \frametitle{Creating Variables with {\tt :=}}
  {\tt:=} is a shortcut for creating one or more new type-inferred
  variables in a function.  When setting multiple variables with
  {\tt:=} at least one of the variables on the left-hand side of the
  expression must be new.  When using {\tt :=} inside of a block, it
  will prefer to create shadow variables rather than re-assign
  variables inherited from the enclosing context.
\begin{lstlisting}[language=Golang]
x, y := 1, 2
{
  x, z := 3, 3
  fmt.Println(x, y, z) // 3 2 3
}
fmt.Println(x, y) // 1 2
\end{lstlisting}
\end{frame}

\begin{frame}
  \frametitle{Assignment}
  A variable can be assigned any number of times during execution.
  Rules of assignment are:
  \begin{itemize}
  \item You can only assign a variable that exists
  \item You can only assign a value of a compatible type
  \end{itemize}
\end{frame}

\subsection{Pointers}
\begin{frame}
  \frametitle{About Pointers}
  A pointer represents the machine address of a variable.  A pointer
  represents that address in the processes virtual memory space that
  contains the start of the data stored in the area of memory.
\end{frame}

\begin{frame}
  \frametitle{Pointer Types}
  A \emph{pointer type} is a type that represents a pointer to a value
  of the underlying type.  Pointer types in Go are represented with
  {\bf *}.  A pointer can be \emph{nil}, or it can be assigned the
  value of the address of another variable.  The {\bf \&} operator
  allows you to get the memory address of a variable.  You cannot take
  the address of a constant.  You can create a new pointer \emph{new}.
\end{frame}

\begin{frame}[fragile]
  \frametitle{Creating Pointers}
  \begin{tabular}{| l | l | p{2in} |}
    \hline
    {\bf type} & {\bf pointer type} & {\bf example} \\
    \hline
    {\tt string} & {\tt *string} &
\begin{verbatim}y := "foo"; x := &y\end{verbatim}
    \\
    \hline
    {\tt int} & {\tt *int} &
\begin{verbatim}x := new(int); *x = 6\end{verbatim}
    \\
    \hline
    {\tt Foo} & {\tt *Foo} &
\begin{verbatim}type Foo struct {
  A int
  B int
}
x := &Foo{B: 7}\end{verbatim}
    \\
    \hline
  \end{tabular}
\end{frame}

\subsection{Arrays and Slices}

\begin{frame}
  \frametitle{Arrays and Slices}
  An array in go is a fixed-size set of values stored in a contiguous
  area of memory.  A slice is a variable-sized collection of elements
  that uses arrays internally to manage the data.  Arrays and slices
  in Go are 0-indexed and both share similar syntax.

  Unlike C and C++, an array in Go is a value type, if you want to
  pass an array by reference, you need to explicitly use a pointer
  type.
\end{frame}

\begin{frame}[fragile]
  \frametitle{Arrays and Slices Example}
\begin{lstlisting}[language=Golang]
x := [2]int{1, 2}   // x is an array
y := []int{}        // y is an empty slice
z := make([]int, 3) // z is {0,0,0}
\end{lstlisting}
\end{frame}

\begin{frame}
  \frametitle{Iterating Over Slices}
  It's common to iterate over elements in an array or slice.  Go
  offers a special type of for-loop that will allow you to loop over
  the elements of a slice or array.  The {\tt range} keyword will
  return one or two values during each iteration of the loop.  The
  first value will be the current index.  The second returned value,
  if specified, will be the value of the array or slice at that
  location.
\end{frame}

\begin{frame}[fragile]
  \frametitle{Iteration Example}
\begin{lstlisting}[language=Golang]
x := []int{0, 1, 2, 3, 4}
for idx, val := range x {
  fmt.Printf("x[%d] = %d; ", idx, val)
}
\end{lstlisting}
\begin{verbatim}
x[0] = 0; x[1] = 1; x[2] = 2; x[3] = 3; x[4] = 4;
\end{verbatim}
\end{frame}

\begin{frame}[fragile]
  \frametitle{Common Slice Operations}
  \begin{tabular}{| l | p{3in} |}
    \hline
    Function  & Example \\
    \hline
    \emph{append} &
\begin{verbatim}x := []int{}; x = append(x, 1};\end{verbatim}
    \\
    \hline
    \emph{len} &
\begin{verbatim} x := []int{1,2,3}; y := len(x)\end{verbatim}
    \\
    \hline
    {\bf :} &
\begin{verbatim} x := []int{1,2,3,4}; y := x[1:3]\end{verbatim}
    \\
    \hline
  \end{tabular}
\end{frame}

\subsection{Functions}

\begin{frame}
  \frametitle{Functions}
  Functions in Go start with the keyword \emph{func}.  A function make
  take as input zero or more values, and may return zero \emph{or
    more} values.  Functions in golang may return multiple values.

  It is common in go for functions that may fail to return both a
  value and an error.
\end{frame}

\begin{frame}[fragile]
  \frametitle{Function Example}
\begin{lstlisting}[language=Golang]
func CanFail(in int) (int, error) {
  if in < 10 {
    return 0, errors.New("out of range")
  }
  return (in + 1), nil
}
\end{lstlisting}
\end{frame}

\begin{frame}
  \frametitle{First Class Functions}
  Functions are first-class types in Go.  You can assign a function to
  a variable, use it as an argument to a function, or return one from
  a function.
\end{frame}

\begin{frame}[fragile]
  \frametitle{Function Example}
\begin{lstlisting}[language=Golang]
func getFunc(s string) (func(int) bool, error) {
        even := func(i int) bool { return 0 == i%2 }
        odd := func(i int) bool { return !even(i) }
        if s == "even" {
                return even, nil
        } else if s == "odd" {
                return odd, nil
        }
        return nil, errors.New("invalid function name")
}
func main() {
        if f, err := getFunc("even"); err == nil {
                fmt.Println(f(3))
        }
        if f, err := getFunc("steven"); err == nil {
                fmt.Println(f(3))
        }
}
\end{lstlisting}
\end{frame}

\subsection{Packages}

\section{Structs and Interfaces}

\section{Concurrency}

\section{Cross Compiling}

\section{Tooling}

\section{Questions?}

\end{document}
