% Copyright 2017 Rebecca Skinner
%
% This work is licensed under the Creative Commons
% Attribution-ShareAlike 4.0 International License. To view a copy of
% this license, visit http://creativecommons.org/licenses/by-sa/4.0/
% or send a letter to Creative Commons, PO Box 1866, Mountain View, CA
% 94042, USA.
\documentclass{beamer}

\title{Introducing Go}
\subtitle{A Brief Survey of The Go Programming Language}
\author{Rebecca Skinner}
\institute{Rackspace Hosting}
\date{\today}

\mode<presentation> {\usetheme{metropolis}}

\usepackage[english]{babel}
\usepackage{times}
\usepackage[T1]{fontenc}
\usepackage{hyperref}
\usepackage{listings}
\usepackage{listings-golang}
\usepackage{color}
\usepackage{amsmath}
\usepackage{csquotes}
\usepackage{verbatim}
\usepackage{fontspec}
\usepackage{pbox}

\definecolor{comment}{rgb}{145,175,188}
\definecolor{keyword}{rgb}{157,163,199}
\definecolor{string}{rgb}{155,204,174}

\lstset{ % add your own preferences
  basicstyle=\small,
  showspaces=false,
  showtabs=false,
  numbers=none,
  numbersep=5pt,
  showstringspaces=false,
  stringstyle=\color[rgb]{0.16, .47, 0},
  tabsize=1
}

\newcommand{\chref}[3] {
  {\color{#1} \href{#2}{\underline{#3}}}
}

\AtBeginSection[]{
  \begin{frame}
    \vfill
    \centering
    \begin{beamercolorbox}[sep=8pt,center,shadow=true,rounded=true]{title}
      \usebeamerfont{title}\insertsectionnumber \\ \insertsectionhead\par%
    \end{beamercolorbox}
    \vfill
  \end{frame}
}

\AtBeginSubsection[]{
  \begin{frame}
    \vfill
    \centering
    \begin{beamercolorbox}[sep=8pt,center,shadow=true,rounded=true]{title}
      \usebeamerfont{title}\insertsectionnumber.\insertsubsectionnumber\\\insertsubsectionhead\par%
    \end{beamercolorbox}
    \vfill
  \end{frame}
}

\begin{document}
\begin{frame}
  \titlepage{}
  \begin{center}
    \small{\chref{blue}{http://creativecommons.org/licenses/by-sa/4.0/}{LICENSE}}
  \end{center}
\end{frame}

\section{What is Go}

\begin{frame}
  \frametitle{History of Go}
  Go was announced in 2009 by google.  It is a compiled, statically
  typed, garbage collected language in the C family.  Go was designed
  to provide modern memory management, concurrency models, and
  networking in a lightweight, readable, and performant language.
\end{frame}

\begin{frame}
  \frametitle{Go Today}
  Go has been steadily gaining a foothold.  Although originally
  designed to be a useful systems language, it has not gained large
  scale adoption there compared to languages like Rust or people
  opting to continue using C and C++.  Go's largest inroads have been
  in devops tooling, such as Kubernetes, Terraform, and Telegraf,
  well as in web application backends and middleware.
\end{frame}

\section{The Basics}

\begin{frame}
  \frametitle{10,000 Foot Overview}
  At a very high level Go, as a language, can be described with the
  following features and design choices:
  \begin{itemize}
  \item Compiled
  \item Type-Inferred
  \item Garbage Collected
  \item Statically Typed
  \item Concurrent
  \item Opinionated
  \item Single-Dispatch Object Oriented
  \end{itemize}
\end{frame}

\subsection{Building Go Programs}

\begin{frame}[fragile]
  \frametitle{The \emph{go} Tool}
  The \emph{go} tool is the main way to build go applications. The
  \emph{go} tool has subcommands for most things you'll do as a
  development when building applications.

  \begin{figure}
\begin{lstlisting}
user@host$ go build
\end{lstlisting}
    \caption{building a go application}
  \end{figure}
\end{frame}

\begin{frame}[fragile]
  \frametitle{Common Go Tools}
  \begin{figure}
  \begin{tabular}{|l|p{3in}|}
    \hline
    \emph{go get} & download and install a package and its dependencies \\
    \hline
    \emph{go build} & Build a project \\
    \hline
    \emph{go test} & Run unit tests \\
    \hline
    \emph{go run} & Run a go source file \\
    \hline
    \emph{go generate} & Generate go source files \\
    \hline
    \emph{go doc} & Show documentation for a given package \\
    \hline
  \end{tabular}
  \caption{Common go tools and their behaviors}
  \end{figure}
\end{frame}

\begin{frame}
  \frametitle{The Go Build Environment}
  Go uses environment variables to define paths to its build tools
  and source files.  For *nix systems, the default system installation
  path is \emph{/usr/local/go}.  If Go is installed elsewhere on the
  system, you must set the \emph{GOROOT} environment variable.

  Go projects for individual users must be stored in the users
  \emph{GOPATH}.  There is no default value for this and it must be
  set per user.  Individual projects must be kept at
  \emph{GOROOT/src/fully-qualified-package-path}.  We'll talk about
  fully qualified package paths later, but it might be something like
  \emph{github.com/rcbops/ops-fabric/utils/go-tool}.
\end{frame}

\begin{frame}[fragile]
  \frametitle{The Development Environment}
  \begin{figure}
  \begin{tabular}{|l|p{3in}|}
    \hline
    \emph{Variable} & \emph{Description} \\
    \hline
    {\tt GOPATH} & \small{Path to go binaries and source for the current user} \\
    \hline
    {\tt GOROOT} & \small{System-wide path to go binaries and packages} \\
    \hline
    {\tt GOARCH} & \small{Target architecture to use when compiling} \\
    \hline
    {\tt GOOS} & \small{Target operating system to use when compiling}\\
    \hline
  \end{tabular}
  \caption{Environment Variables used by Go}
  \end{figure}
\end{frame}

\subsection{Variables}

\begin{frame}
  \frametitle{Variables in Go}
  Variables in go work much like they do in other procedural and
  object oriented languages.  Variables are mutable unless defined as
  constant.  Go differentiates between creating and assigning
  variables.  Creating a variable that already exists, or assigning
  one that doesn't, are both errors.
\end{frame}

\begin{frame}
  \frametitle{Scope}
  Variables in go are block-scoped and closed over the enclosed
  contexts.  A variable in an inner context may shadow a variable in
  the outer context.
\end{frame}

\begin{frame}[fragile]
  \frametitle{Exporting Variables}
  Go uses the concept of \emph{exported} and \emph{unexported}
  variables.  Variables are exported or unexported at the package
  level.  Variable names start with a capital letter are exported, and
  accessible outside of the current package.

\begin{lstlisting}[language=Golang]
package foo

var Exported = 12
var unexported = "foo"
\end{lstlisting}
\end{frame}

\begin{frame}
  \frametitle{Creating Variables}
  Go provides two methods for creating variables.  The {\tt var}
  keyword allows you to define one or more variables in a function or
  package.  The {\tt :=} operator allows to define and initialize a
  new variable inside of a function.
\end{frame}

\begin{frame}
  \frametitle{Default Values}
  Go variables have a default ``zero'' value.  Any newly created
  variables not explicitly given a value default to the zero value.
  The default value of a struct is a struct where each field is set to
  its default value.  The default value of a pointer is {\tt nil}.
\end{frame}

\begin{frame}
  \frametitle{Unused Variables}
  The special variable {\bf \_} allows you to throw away a variable.
  Since Go requires you to always acknowledge return values of
  functions, this is a useful way to explicitly a result from a
  computation.
\end{frame}

\begin{frame}[fragile]
  \frametitle{Creating Variables with {\tt var}}
  The var keyword can be used inside or outside of a function to
  create one or more variables.
\begin{lstlisting}[language=Golang]
var x int
var (
  foo  = 10
  bar  = 90.7
  baz  = "threeve!"
  buzz string
)
\end{lstlisting}
\end{frame}

\begin{frame}[fragile]
  \frametitle{Creating Variables with {\tt :=}}
  {\tt:=} is a shortcut for creating one or more new type-inferred
  variables in a function.  When setting multiple variables with
  {\tt:=} at least one of the variables on the left-hand side of the
  expression must be new.  When using {\tt :=} inside of a block, it
  will prefer to create shadow variables rather than re-assign
  variables inherited from the enclosing context.
\begin{lstlisting}[language=Golang]
x, y := 1, 2
{
  x, z := 3, 3
  fmt.Println(x, y, z) // 3 2 3
}
fmt.Println(x, y) // 1 2
\end{lstlisting}
\end{frame}

\begin{frame}
  \frametitle{Assignment}
  A variable can be assigned any number of times during execution.
  Rules of assignment are:
  \begin{itemize}
  \item You can only assign a variable that exists
  \item You can only assign a value of a compatible type
  \end{itemize}
\end{frame}

\subsection{Pointers}
\begin{frame}
  \frametitle{About Pointers}
  A pointer represents the machine address of a variable.  A pointer
  represents that address in the processes virtual memory space that
  contains the start of the data stored in the area of memory.
\end{frame}

\begin{frame}
  \frametitle{Pointer Types}
  A \emph{pointer type} is a type that represents a pointer to a value
  of the underlying type.  Pointer types in Go are represented with
  {\bf *}.  A pointer can be \emph{nil}, or it can be assigned the
  value of the address of another variable.  The {\bf \&} operator
  allows you to get the memory address of a variable.  You cannot take
  the address of a constant.  You can create a new pointer \emph{new}.
\end{frame}

\begin{frame}[fragile]
  \frametitle{Creating Pointers}
  \begin{tabular}{| l | l | p{2in} |}
    \hline
    {\bf type} & {\bf pointer type} & {\bf example} \\
    \hline
    {\tt string} & {\tt *string} &
\begin{verbatim}y := "foo"; x := &y\end{verbatim}
    \\
    \hline
    {\tt int} & {\tt *int} &
\begin{verbatim}x := new(int); *x = 6\end{verbatim}
    \\
    \hline
    {\tt Foo} & {\tt *Foo} &
\begin{verbatim}type Foo struct {
  A int
  B int
}
x := &Foo{B: 7}\end{verbatim}
    \\
    \hline
  \end{tabular}
\end{frame}

\subsection{Arrays and Slices}

\begin{frame}
  \frametitle{Arrays and Slices}
  An array in go is a fixed-size set of values stored in a contiguous
  area of memory.  A slice is a variable-sized collection of elements
  that uses arrays internally to manage the data.  Arrays and slices
  in Go are 0-indexed and both share similar syntax.

  Unlike C and C++, an array in Go is a value type, if you want to
  pass an array by reference, you need to explicitly use a pointer
  type.
\end{frame}

\begin{frame}[fragile]
  \frametitle{Arrays and Slices Example}
\begin{lstlisting}[language=Golang]
x := [2]int{1, 2}   // x is an array
y := []int{}        // y is an empty slice
z := make([]int, 3) // z is {0,0,0}
\end{lstlisting}
\end{frame}

\begin{frame}
  \frametitle{Iterating Over Slices}
  It's common to iterate over elements in an array or slice.  Go
  offers a special type of for-loop that will allow you to loop over
  the elements of a slice or array.  The {\tt range} keyword will
  return one or two values during each iteration of the loop.  The
  first value will be the current index.  The second returned value,
  if specified, will be the value of the array or slice at that
  location.
\end{frame}

\begin{frame}[fragile]
  \frametitle{Iteration Example}
\begin{lstlisting}[language=Golang]
x := []int{0, 1, 2, 3, 4}
for idx, val := range x {
  fmt.Printf("x[%d] = %d; ", idx, val)
}
\end{lstlisting}
\begin{verbatim}
x[0] = 0; x[1] = 1; x[2] = 2; x[3] = 3; x[4] = 4;
\end{verbatim}
\end{frame}

\begin{frame}[fragile]
  \frametitle{Common Slice Operations}
  \begin{tabular}{| l | p{3in} |}
    \hline
    Function  & Example \\
    \hline
    \emph{append} &
\begin{verbatim}x := []int{}; x = append(x, 1};\end{verbatim}
    \\
    \hline
    \emph{len} &
\begin{verbatim} x := []int{1,2,3}; y := len(x)\end{verbatim}
    \\
    \hline
    {\bf :} &
\begin{verbatim} x := []int{1,2,3,4}; y := x[1:3]\end{verbatim}
    \\
    \hline
  \end{tabular}
\end{frame}

\subsection{Maps}

\begin{frame}
  \frametitle{Golang Maps}
  Go provides maps as a fundamental data type.  Maps may be keyed on
  any comparable values, and can have values of any Go type.  The go
  runtime randomizes maps in order to surface bugs caused by relying
  on the stability of the internal hashing algorithm.  The built-in
  map types support concurrent reads, but do not support conccurrent
  writes, or read/write concurrency.
\end{frame}

\begin{frame}[fragile]
  \frametitle{Creating Maps}
\begin{lstlisting}[language=Golang]
map1 := make(map[string]string, 10)
map2 := map[string]string{
  "foo":  "bar",
  "fizz": "buzz",
}
\end{lstlisting}
\end{frame}

\begin{frame}
  \frametitle{Accessing Elements}
  Elements of a map can be accessed using brackets, as with many other
  languages.  Accessing a member of a map will return one or two
  values.  The first value will be the value of the map at the
  specified key, or the zero value of the type if the element doesn't
  exist.  The optional second return value is a bool which will be set
  to \emph{true} if the value was found, and \emph{false} if it
  wasn't.
\end{frame}

\begin{frame}[fragile]
  \frametitle{Accessing Elements}
\begin{lstlisting}[language=Golang]
var (
   n  int
   ok bool
   m  = map[int]int{0: 0, 2: 2, 4: 4}
)
n = m[0]     // 0
n = m[1]     // 0
n, ok = m[1] // 0, false
n, ok = m[2] // 2, true
\end{lstlisting}
\end{frame}

\begin{frame}
  \frametitle{Iterating Over Maps}
  The builtin \emph{range} function allows you to iterate over keys,
  and optionally values, of a map.  The syntax for iterating over maps
  is the same as it is for arrays and slices.
\end{frame}

\begin{frame}[fragile]
  \frametitle{Iterating Over Maps}
\begin{lstlisting}[language=Golang]
func main() {
  m := map[string]int{"foo": 0, "bar": 1, "baz": 2}
  for k, v := range m {
    fmt.Printf("%s => %d\n", k, v)
  }
}
\end{lstlisting}
\end{frame}

\subsection{Functions}

\begin{frame}
  \frametitle{Functions}
  Functions in Go start with the keyword \emph{func}.  A function make
  take as input zero or more values, and may return zero \emph{or
    more} values.  Functions in golang may return multiple values.

  It is common in go for functions that may fail to return both a
  value and an error.
\end{frame}

\begin{frame}[fragile]
  \frametitle{Function Example}
\begin{lstlisting}[language=Golang]
func CanFail(in int) (int, error) {
  if in < 10 {
    return 0, errors.New("out of range")
  }
  return (in + 1), nil
}
\end{lstlisting}
\end{frame}

\begin{frame}
  \frametitle{First Class Functions}
  Functions are first-class types in Go.  You can assign a function to
  a variable, use it as an argument to a function, or return one from
  a function.
\end{frame}

\begin{frame}[fragile]
  \frametitle{Function Example}
\begin{lstlisting}[language=Golang]
func getFunc(s string) (func(int) bool, error) {
  even := func(i int) bool { return 0 == i%2 }
  odd  := func(i int) bool { return !even(i) }
  if s == "even" {
    return even, nil
  } else if s == "odd" {
    return odd, nil
  }
  return nil, errors.New("invalid function name")
}
func main() {
  if f, err := getFunc("even"); err == nil {
    fmt.Println(f(3))
  }
  if f, err := getFunc("steven"); err == nil {
    fmt.Println(f(3))
  }
}
\end{lstlisting}
\end{frame}

\subsection{Packages}

\begin{frame}
  \frametitle{Packages}
  Packages are the basic unit of modularity in go applications.
  Executables should have a package called \emph{main} at the top
  level of your project directory.

  Packages contain one or more files, should be named after the
  directory that contains the files.
\end{frame}

\begin{frame}
  \frametitle{Package Paths}
  For packages that are not relative to the current working directory,
  packages are given by their full path relative to \emph{GOROOT/src}
  or \emph{PROJECTROOT/vendor}.  Typically, the path reflects the URL
  you would use to \emph{go get} a package, so for example a github
  project hosted at \url{https://github.com/rebeccaskinner/converge}
  containing a package, \emph{resource} would be stored on disk at
  \emph{GOROOT/src/github.com/rebeccaskinner/converge/resource} and
  the import path would be
  \emph{``github.com/rebeccaskinner/converge/resource''}
\end{frame}

\section{Structs and Interfaces}

\begin{frame}[fragile]
  \frametitle{Structs}
  A struct is a named structured data type.  The fields in a struct
  may be exported or unexported.  The exported fields are accessible
  from outside of package where the struct is defined.  Within the
  package all fields of a struct are accessible.  Structs are regular
  values that can be created like any other type of variable.  You can
  set specific fields of a struct using key/value syntax.

\begin{lstlisting}[language=Golang]
type Example struct {
  Foo int
  Bar string
}

e := Example{Foo: 0, Bar: "bar"}
\end{lstlisting}
\end{frame}

\begin{frame}[fragile]
  \frametitle{Empty Structs}
  Empty structs do not take up any memory at runtime.  This allows
  them to be used in a way analagous to symbols in languages like lisp
  and ruby.  Empty structs are written \verb!struct{}!.  It's common
  to use empty structs with maps when you are only concerned with
  presence in a set.
\end{frame}


\begin{frame}[fragile]
  \frametitle{Embedded Structs}
  Structs may be embedded into one another.  When structs are
  embedded, they inherit the embedded structs fields.  The struct
  itself may be accessed by its type name.  To created an embedded
  struct, add it into the containing struct unnamed.
\begin{lstlisting}[language=Golang]
type Embedded struct {
  Val int
}
type Example struct {
  Embedded
  Foo int
  Bar string
}
func main() {
  e := Example{Embedded: Embedded{Val: 0}}
  fmt.Println(e.Val)
}
\end{lstlisting}
\end{frame}

\begin{frame}
  \frametitle{Receivers}
  Receivers are functions that are attached to a struct.  This is Go's
  solution to object-oriented programming.  Receivers may be attached
  to structs, or pointers to structs, where they are called
  \emph{pointer receivers}.  Functions attached to structs may be
  called on a specific instance of a struct with \emph{.MethodName()},
  this should look familliar if you've used languages like C++, Java,
  or Javascript.
\end{frame}

\begin{frame}[fragile]
  \frametitle{Receivers}
\begin{lstlisting}[language=Golang]
type Example struct {
  Foo int
  Bar string
}
func (e *Example) String() string {
return fmt.Sprintf("%s: %d", e.Bar, e.Foo)
}
func main() {
  e := &Example{Foo: 0, Bar: "bar"}
  fmt.Println(e.String())
}
\end{lstlisting}
\end{frame}

\begin{frame}[fragile]
  \frametitle{Receivers On Embedded Structs}
  The receivers attached to embedded structs are accessible from
  instances of the embedding function.  If both the embedding and
  embedded structures have receivers for the same function, the
  embedding structures version will be called.
\begin{lstlisting}[language=Golang]
type Embedded struct{}
func (e *Embedded) EmbeddedFunc() string {
return "foo"
}
type Example struct{ Embedded }
func main() {
        e := &Example{}
        fmt.Println(e.EmbeddedFunc())
}
\end{lstlisting}
\end{frame}

\begin{frame}
  \frametitle{Interfaces}
  An interface defines a set of functions.  A structure implements an
  interface if it has receivers of the correct name and type for each
  of the functions listed by the interface.  It's important to note
  that interface fulfillment is automatic.  You may define an
  interface in one package that is automatically fulfilled by some
  type imported from some other package.  This is very useful for
  mocking for unit tests, creating factories, or otherwise proxying
  types.
\end{frame}

\begin{frame}[fragile]
  \frametitle{Interfaces}
\begin{lstlisting}[language=Golang]
type Thinger interface {
  DoTheThing(int) string
}
func ShowTheThing(t Thinger) {
  fmt.Println(t.DoTheThing(4))
}
type Example struct{}
func (e *Example) DoTheThing(i int) string { return fmt.Sprint(i) }
func main() {
  ShowTheThing(&Example{})
}
\end{lstlisting}
\end{frame}

\begin{frame}
  \frametitle{A Bit More On Interfaces}
  Like structs, interfaces may be nested.  It is idiomatic in Go to
  create many small interfaces with one or two functions, and compose
  them into larger interfaces.

  Idiomatically, Go functions should accept interfaces as input, and
  return concrete types as output.
\end{frame}

\begin{frame}
  \frametitle{The Empty Interface}
  The empty interface is represented in Go as \emph{$interface\{\}$}.
  The empty interface can represent any type, and is often used in
  place of generics when trying to create libraries with containers.
\end{frame}


\begin{frame}[fragile]
  \frametitle{Casting Interfaces}
  One common challenge when working with Go applications is the need
  to go from an interface.  Type casting in go is not limited to
  interfaces, but it shown here since this is by far the most common
  use-case.

  A type cast returns two values, the result of the cast, and a bool
  indicating whether the types were compatible.  To cast a value, call
  \verb!value.(newtype)!
\end{frame}

\begin{frame}[fragile]
  \frametitle{Casting Example}
\begin{lstlisting}[language=Golang]
type Thinger interface {
        DoTheThing(int) string
}
func asThinger(t Thinger) Thinger { return t }
type Example struct{}
func (e *Example) ExampleFunc() {}
func (e *Example) DoTheThing(i int) string { return fmt.Sprint(i) }
func main() {
        thinger := asThinger(&Example{})
        example, ok := thinger.(*Example)
        if !ok {
                return
        }
        example.ExampleFunc()
}
\end{lstlisting}
\end{frame}

\section{Concurrency}

\begin{frame}
  \frametitle{Concurrency}
  Go is built around a message passing concurrency system with support
  for lightweight threads.  Go uses channels for passing messages
  between concurrent routines.
\end{frame}

\begin{frame}
  \frametitle{Channels}
  A channel is a way of passing messages from within or between go
  routines.  Channels are first class values and can be created,
  passed into and returned from functions, stored in maps, and
  compared.

  A channel may be designated for input, output, or both, and has a
  size and a type.  Input channels may only be written to, output
  channels may only be read from.  The only values that can be written
  to, or read from, a channel are given by its type.

  The size of a channel represents the total number of items it can
  buffer before new writes will block.
\end{frame}

\begin{frame}[fragile]
  \frametitle{Creating a Channel}
  Channels are created with \emph{make} and are designated with the
  keyword \emph{chan}.  Arrows are used to indicate whether a channel
  is read-only, write-only, or read-write.
\end{frame}

\begin{frame}[fragile]
  \frametitle{Channel Types}
  \begin{tabular}{|p{2in}|p{1.5in}|}
    \hline
    \verb!chan int! & a read/write channel containing integers \\
    \hline
    \verb!chan<- string! & a write-only channel containing strings \\
    \hline
    \verb!<-chan (chan<- int)! & a read-only channel that returns write-only channels containing integers \\
    \hline
  \end{tabular}
\end{frame}

\begin{frame}
  \frametitle{Channel Compatability}
  Channels will automatically downcast to channels of more limited
  permissions.  In other words, you may use a read-write channel where
  a read-only or write-only channel is expected, but not vice-versa.
\end{frame}

\begin{frame}[fragile]
  \frametitle{Channel IO}
  The same arrows used to describe a channel as read-only or
  write-only are also used to read from or write to a channel.
\begin{lstlisting}[language=Golang]
ch := make(chan int, 1)
ch <- 1 // write the value 1 to the channel
val := <-ch // read the value 1 from the channel
\end{lstlisting}
\end{frame}


\begin{frame}[fragile]
  \frametitle{Go Routines}
  Go routines are lightweight asynchronous threads that run a specific
  function.  Go routines will run until the function they are
  executing returns, or the application terminates.  Execution of
  go-routines is nondeterministic.  Goroutines are spawned with they
  keyword \emph{go}

  \verb! go func(){ fmt.Println("hello")}()!
\end{frame}

\begin{frame}[fragile]
  \frametitle{Defer}
  Defer is a useful way of handling something that will occur when a
  function exits.  Defer will cause a funciton to be executed after
  the final statement in the containing function, but before the value
  is returned to the caller.  Defer can be used to free up resources
  or to signal to another go-routine that it has exited.

  \verb!defer func(){ fmt.Println("exiting") }()!
\end{frame}

\begin{frame}
  \frametitle{Select}
  Select allows the user to take an action based on selecting from
  some group of blocking calls.  Select statements are most often used
  to manage control flow by coordinating actions based on input from
  channels being written to from various go routines.
\end{frame}

\begin{frame}[fragile]
  \frametitle{Select Example}
\begin{lstlisting}[language=Golang]
func main() {
  ch := make(chan int, 1)
  done := make(chan struct{}, 1)
  go func() {
    for idx := 0; idx < 10; idx++ {ch <- idx}
    done <- struct{}{}
  }()
  for {
    select {
    case val := <-ch:
      fmt.Println(val)
    case <-done:
      return
    }
  }
}
\end{lstlisting}
\end{frame}


\section{Questions?}

\end{document}
