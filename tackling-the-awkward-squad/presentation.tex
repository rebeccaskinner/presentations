% Copyright 2017 Rebecca Skinner
%
% This work is licensed under the Creative Commons
% Attribution-ShareAlike 4.0 International License. To view a copy of
% this license, visit http://creativecommons.org/licenses/by-sa/4.0/
% or send a letter to Creative Commons, PO Box 1866, Mountain View, CA
% 94042, USA.
\documentclass{beamer}

\title{Tackling ``Tackling the Awkward Squad''}
\subtitle{}
\author{Rebecca Skinner\\ \small{@cercerilla}}
\institute{Papers We Love St. Louis}
\date{\today}

\mode<presentation> {\usetheme{metropolis}}

\usepackage[english]{babel}
\usepackage{times}
\usepackage[T1]{fontenc}
\usepackage{hyperref}
\usepackage{listings}
\usepackage{color}
\usepackage{amsmath}
\usepackage{csquotes}
\usepackage{verbatim}
\usepackage{fontspec}
\usepackage{pbox}

\definecolor{comment}{rgb}{145,175,188}
\definecolor{keyword}{rgb}{157,163,199}
\definecolor{string}{rgb}{155,204,174}

\lstset{ % add your own preferences
  basicstyle=\tiny,
  showspaces=false,
  showtabs=false,
  numbers=none,
  numbersep=5pt,
  showstringspaces=false,
  stringstyle=\color[rgb]{0.16, .47, 0},
  tabsize=1
}

\newcommand{\chref}[3] {
  {\color{#1} \href{#2}{\underline{#3}}}
}

\AtBeginSection[]{
  \begin{frame}
    \vfill
    \centering
    \begin{beamercolorbox}[sep=8pt,center,shadow=true,rounded=true]{title}
      \usebeamerfont{title}\insertsectionnumber \\ \insertsectionhead\par%
    \end{beamercolorbox}
    \vfill
  \end{frame}
}

\AtBeginSubsection[]{
  \begin{frame}
    \vfill
    \centering
    \begin{beamercolorbox}[sep=8pt,center,shadow=true,rounded=true]{title}
      \usebeamerfont{title}\insertsectionnumber.\insertsubsectionnumber\\\insertsubsectionhead\par%
    \end{beamercolorbox}
    \vfill
  \end{frame}
}

\begin{document}
\begin{frame}
  \titlepage{}
  \begin{center}
    \small{\chref{blue}{http://creativecommons.org/licenses/by-sa/4.0/}{LICENSE}}
  \end{center}
\end{frame}

\begin{frame}
  \frametitle{The Paper}
  \begin{center}
    \emph{Tackling the Awkward Squad: monadic input/output, concurrency, exceptions, and foreign-language calls in Haskell}
    \vfill
    Simon Peyton Jones\\
    \emph{Microsoft Research, Cambridge}\\
    April 7, 2010
    \vfill
\chref{blue}{https://www.microsoft.com/en-us/research/wp-content/uploads/2016/07/mark.pdf}{link}
\end{center}
\end{frame}

\section{Introduction}
\begin{frame}
  \frametitle{About the Paper}
  Tackling The Awkward Squad looks at the real problems and solutions
  for writing applications in haskell.  It examines how monads, in
  particular, arose as a practical solution to real problems faced by
  language designers.
\end{frame}

\begin{frame}
  \frametitle{Understanding the Awkward Squad}
  The awkward squad is the name given to a few concepts that had
  proven difficult to implement with pure lazy semantics of languages
  like haskell.  In particular the awkward squad consists of:
  \begin{itemize}
    \item Input and Output
    \item Concurrency
    \item Exception Handling
    \item Interfacing with Other Languages
  \end{itemize}
\end{frame}
\begin{frame}
  \frametitle{What We'll Cover}
  The paper discusses both the language level constructs to tackle the
  awkward squad, as well as the denotional and operational semantics
  used to formalize the features for compiler implementers.  In this
  presentation we'll focus on the lingustic constructs and eschew the
  denotional and operational semantics of interest to compiler
  implementers.

  Futhermore, some specific areas of the paper will be omitted where
  there has been futher research that has advanced the state of the
  art beyond what was known at the time that this paper was written.
\end{frame}

\begin{frame}
  \frametitle{Why This is Cool- Even If You Don't Like Haskell}
  The awkward squad is not only a set of problems faced by the
  implementers of lazy functional languages- these are areas where
  practical work can be accomplished in all languages to simplify
  program design and reduce errors.  Understanding the approaches
  taken by language designers can have a material impact on how we
  address these concerns in other languages as well.
\end{frame}

\section{Input and Output}
\begin{frame}[fragile]
\begin{center}
\begin{lstlisting}
primes :: [Integer]
primes = sieve [2..]
  where
sieve (p:xs) = p : sieve [x | x <- xs, x `mod` p > 0 ]
\end{lstlisting}
\end{center}
\end{frame}

\begin{frame}
  \frametitle{Solipsism}
  \emph{Solipsism} is the position in Metaphysics and Epistemology
  that the mind is the only thing that can be known to exist and that
  knowledge of anything outside the mind is unjustified.
\end{frame}

\begin{frame}
  \frametitle{Solipsistic Applications}
  A solipsistic application is a rather useless thing, unless we are
  trying to heat a cold room.  Without the ability to interact with
  the outside world a program cannot be of actual use.  This
  interaction requires receiving \emph{input} from the outside world,
  and providing \emph{output} to the world about the state or result
  of the program.
\end{frame}

\begin{frame}
  \frametitle{The Problem of IO}
  Input and Output (IO) can the the cause of some consternation for a
  pure lazy language.  In the next few slides we'll take a look at
  some specific issues that the paper raises regarding IO in languages
  like haskell.
\end{frame}

\begin{frame}
  \frametitle{Lazy Evaluation?}
  Lazy evaluation complicates the notion of input and output because
  there is no concept of strict ordering for when input and output
  should occur.
\end{frame}

\begin{frame}[fragile]
  \frametitle{The Problem of Sequencing}
  Much of the problem of lazy evaluation is also one of sequencing.  Consider the program snippet below:
\begin{lstlisting}
results = [ print "hello", print "world", print "!" ]
\end{lstlisting}
  In this example, there is no clear way for us to understand which
  order the print statements should be evaluated in a lazy language.
\end{frame}

\begin{frame}[fragile]
  \frametitle{A Change in the World}
  Input and output are problematic not just from a lazy evaluation and
  sequencing standpoint- input and output also make functions impure
  because they rely on external effects.  Imagine the following
  pseudocode application below which appends a number of 0's to a file
  each call equivalent to half the remaining disk space:

\begin{lstlisting}
fn effectfulFunction: Integer
  let fileSize <- availableDiskSpace / 2
  let fileData <- repeat '0' for fileSize
  appendFile "/tmp/log" fileData
  return fileSize
\end{lstlisting}

  Each call to this subroutine will affect the external state of the
  machine in a way that means that we cannot treat this as a pure
  function.
\end{frame}
\begin{frame}
  \frametitle{What we Need}
  Solving the problem of input and output requires two notions:
  \begin{itemize}
  \item A way of keeping track of the state of the world, so that we
    can capture changes to the state of the world, representing side
    effects in a pure way
  \item A way of forcing a specific sequence of evaluation, in order
    to ensure that we are able to represent the dependencies between
    effects.
  \end{itemize}
\end{frame}

\begin{frame}[fragile]
  \frametitle{Monads}
  Monads provided a convenient mathematical notion for how to manage
  both carrying additional context around a value, and a way to
  enforce sequencing on operations in a lazy language.

\begin{lstlisting}
class Monad m where
  return :: a -> m a
  (>>=)  :: m a -> (a -> m b) -> m b
  (>>)   :: m a -> m b -> m b
\end{lstlisting}
\end{frame}

\begin{frame}
  \frametitle{Other Things That Are Secretly IO}
  As we start to pick apart the things that involve a persisent world
  state there are several other patterns that we can observe fall
  under the same umbrellas a basic text input and output:
  \begin{itemize}
  \item looping
  \item mutable variables
  \end{itemize}
\end{frame}

\begin{frame}[fragile]
  \frametitle{Do Notation}
  As we start to look at implementations and use of various features
  that are implemented using the monadic io-actions, we start to
  observe how syntactic sugar will allow us to make the language more
  familiar and easier to update.  Consider:
\begin{lstlisting}
readTwoWriteFour :: IO ()
readTwoWriteFour =
  getChar >>= \fstChr ->
    getChar >>= \sndChar ->
      putChar fstChr >>
        putChar sndChr >>
          putChar fstChr >>
            putChar sncChr
\end{lstlisting}
\end{frame}

\begin{frame}[fragile]
  By introduction the do notation wich adds implicit bind operations
  and lambda creation, we can simplify monadic code into something
  that more intituitvely resembles it's semantics:
\begin{lstlisting}
readTwoWriteFour :: IO
readTwoWriteFour = do
  fstChr <- getChar
  sndChr <- getChar
  putChar fstChar; putChar sndCar
  putChar fstChar; putChar sndCar
\end{lstlisting}
\end{frame}

\section{Concurrency}
\begin{frame}
  \frametitle{Concurrency and Parallelism}
  Before we begin looking at managing concurrency and parallelism
  let's take a brief look at the difference between the two:
  \begin{itemize}
  \item \emph{Parallelism} uses multiple threads of execution (across
    logical or physical cores, processors, or computers) in order to
    increase performance of an algorithm.  Parallelism is an
    implementation detail that does not affect the semantics of the
    application.
  \item \emph{Concurrency} is when multiple independent threads of
    execution are simultaneously executed.  The program may or may not
    be distributed across many individual threads, but the execution
    is non-deterministic and semantically impacts program execution.
  \end{itemize}
\end{frame}
\begin{frame}
  \frametitle{Fork}

\end{frame}
\begin{frame}
  \frametitle{Channels}

\end{frame}
\section{Exception Handling}
\begin{frame}
  \frametitle{Throw and Catch}

\end{frame}
\begin{frame}
  \frametitle{Pure Exceptions}

\end{frame}
\begin{frame}
  \frametitle{IOError}

\end{frame}
\begin{frame}
  \frametitle{Async Exceptions}

\end{frame}
\section{Foreign-Language Calls}
\begin{frame}
  \frametitle{Calling To and From C}

\end{frame}
\begin{frame}
  \frametitle{Dynamic Calls}

\end{frame}

\begin{frame}
  \frametitle{Marshalling}

\end{frame}
\begin{frame}
  \frametitle{Memory Management}

\end{frame}
\begin{frame}
  \frametitle{Foreign Objects}

\end{frame}
\section{Questions?}
\end{document}
