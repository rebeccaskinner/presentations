% Copyright 2017 Rebecca Skinner
%
% This work is licensed under the Creative Commons
% Attribution-ShareAlike 4.0 International License. To view a copy of
% this license, visit http://creativecommons.org/licenses/by-sa/4.0/
% or send a letter to Creative Commons, PO Box 1866, Mountain View, CA
% 94042, USA.
\documentclass{beamer}

\title{Row, Row, Row your Go}
\subtitle{Design Patterns for Structural Polymorphism in Go}
\author{Rebecca Skinner\\ \small{@cercerilla}}
\institute{Target}
\date{\today}

\mode<presentation> {\usetheme{metropolis}}

\usepackage[english]{babel}
\usepackage{times}
\usepackage[T1]{fontenc}
\usepackage{hyperref}
\usepackage{listings}
\usepackage{listings-golang}
\usepackage{color}
\usepackage{amsmath}
\usepackage{csquotes}
\usepackage{verbatim}
\usepackage{fontspec}
\usepackage{pbox}
\usepackage{soul}

\def\checkmark{\tikz\fill[scale=0.4](0,.35) -- (.25,0) -- (1,.7) -- (.25,.15) -- cycle;}

\definecolor{comment}{rgb}{145,175,188}
\definecolor{keyword}{rgb}{157,163,199}
\definecolor{string}{rgb}{155,204,174}

\lstset{ % add your own preferences
  basicstyle=\tiny,
  showspaces=false,
  showtabs=false,
  numbers=none,
  numbersep=5pt,
  showstringspaces=false,
  stringstyle=\color[rgb]{0.16, .47, 0},
  tabsize=1
}

\newcommand{\chref}[3] {
  {\color{#1} \href{#2}{\underline{#3}}}
}

\AtBeginSection[]{
  \begin{frame}
    \vfill
    \centering
    \begin{beamercolorbox}[sep=8pt,center,shadow=true,rounded=true]{title}
      \usebeamerfont{title}\insertsectionnumber \\ \insertsectionhead\par%
    \end{beamercolorbox}
    \vfill
  \end{frame}
}

\AtBeginSubsection[]{
  \begin{frame}
    \vfill
    \centering
    \begin{beamercolorbox}[sep=8pt,center,shadow=true,rounded=true]{title}
      \usebeamerfont{title}\insertsectionnumber.\insertsubsectionnumber\\\insertsubsectionhead\par%
    \end{beamercolorbox}
    \vfill
  \end{frame}
}

\begin{document}
\begin{frame}
  \titlepage{}
  \begin{center}
    \small{\chref{blue}{http://creativecommons.org/licenses/by-sa/4.0/}{LICENSE}}
  \end{center}
\end{frame}

\section{The Polymorphism Problem}
\begin{frame}
  \frametitle{Making Simple Things Easy}
  \begin{quote}
    A good language should make \st{easy} {\bf simple} things easy, and hard things possible.
  \end{quote}
\end{frame}

\begin{frame}
  \frametitle{But For Go}
  \begin{center}
    Go feels like\\\vspace{2mm}
    \begin{tabular}{|c|c|c|} \hline
      & Simple & Complex \\ \hline
      Easy & $\times$ & \checkmark \\ \hline
      Hard & \checkmark & $\times$ \\ \hline
    \end{tabular}
  \end{center}
\end{frame}

\subsection{An Identity Crisis}
\begin{frame}
  \frametitle{Simple. Easy. Impossible}
  The identity function takes a value and returns it.
\end{frame}

\begin{frame}
  \frametitle{Why?}
  The identity function can be very useful when working with
  high-order functions.
\end{frame}

\begin{frame}[fragile]
  \frametitle{Silly Example}
\begin{lstlisting}[language=Golang]
func ReverseMapInt(f func(int) int, ints []int) (out []int) {
        for _, i := range ints {
                out = append([]int{f(i)}, out...)
        }
        return
}
func IntID(i int) int   { return i }
func IntSucc(i int) int { return i + 1 }
func main() {
        ints := []int{1, 2, 3}
        fmt.Println(ReverseMapInt(IntSucc, ints))
        fmt.Println(ReverseMapInt(IntID, ints))
}
\end{lstlisting}
\end{frame}

\begin{frame}[fragile]
  \frametitle{output}
\begin{lstlisting}
-*- mode: compilation; default-directory: "~/go/src/playground/at-2019-05-04-182644/" -*-
Compilation started at Sat May  4 18:28:40

go run *.go
[4 3 2]
[3 2 1]

Compilation finished at Sat May  4 18:28:40
\end{lstlisting}
\end{frame}

\subsection{A Generic Problem}

\begin{frame}
  \frametitle{Problem}
  Write an identity function that works for any input value.
\end{frame}

\begin{frame}[fragile]

\begin{lstlisting}[language=ruby]
# ruby
def id(a) return a; end
\end{lstlisting}
\pause

\begin{lstlisting}[language=python]
# python
from typing import TypeVar, Generic
A = TypeVar('A')
def id(input: A) -> A:
    return input
\end{lstlisting}
\pause

\begin{lstlisting}[language=haskell]
-- haskell
id a = a
\end{lstlisting}
\pause

\begin{lstlisting}
(* ocaml *)
let id a = a;;
\end{lstlisting}
\pause

\begin{lstlisting}
// rust
fn id<A>(input: A) -> A {return input;}
\end{lstlisting}
\pause

\begin{lstlisting}
(* sml *)
fun id(x:'a):'a = x;
\end{lstlisting}
\pause

\begin{lstlisting}
// Javascript
function id(x) { return x; }
\end{lstlisting}
\pause

\begin{lstlisting}
// Typescript
function id<T>(x: T): T { return x; }
\end{lstlisting}
\pause

\begin{lstlisting}[language=Java]
// Java
class Identity {public static <T> T id(T x) {return x; }}
\end{lstlisting}
\end{frame}

\begin{frame}
  \frametitle{And in Go}
    \begin{center}
    \includegraphics[height=.8\paperheight]{img/nopher}
  \end{center}
\end{frame}

\begin{frame}
  \frametitle{Polymorphism}
  Go lacks the ability to directly express this type of polymorphism.
  Our only choices are to avoid this type of constructor, or else to
  fall back to using go generate or working with \verb!interface{}!.
\end{frame}

\section{The Many Shapes of Polymorphism}

\begin{frame}
  \frametitle{Types of Polymorphism}
  The identity examples rely on {\it parametric} and {\it return type}
  polymorphism, which go doesn't support.

  Go doesn't support formally expressing full structural subtyping at
  the type level, but it's approach to handling structure embedding
  and interfaces is semantically similar to structural subtyping.
\end{frame}

\begin{frame}[fragile]
  \frametitle{Structural Embedding as Subtyping}

\end{frame}



\section{Questions?}
\end{document}
